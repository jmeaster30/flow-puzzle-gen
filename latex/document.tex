\documentclass[10pt,a4paper,openany]{article}

\usepackage[utf8]{inputenc}
\usepackage[T1]{fontenc}
\usepackage{amsmath}
\usepackage{amssymb}
\usepackage{graphicx}

\title{Understanding the Puzzles of Flow}
\author{John Easterday}

\begin{document}

\maketitle

\section{Introduction}
%This paragraph can be a little more fleshed out of my motivation
I really like this mobile game called Flow and playing it has piqued my interest in various properties about the puzzles. Flow gameplay is as follows:
	
	
\begin{enumerate}
	\item You are given a set of points on a 2d square grid of cells
	\item Each point is given a color and matches 1 and only 1 other point
	\item The goal is to connect each of the points to it's partner point
	\item The connections may not intersect (cannot occupy the same cell on the grid as another connection)
	\item The connections must fill the grid completely (all cells in the grid contain either a connection or is one of the endpoints)
\end{enumerate}
	
This is a pretty straight forward set of rules but BigDuckGames has challenged some of these rules in other games. For example, Flow Hexes is on a hex grid instead of a square grid and Flow Bridges allows connections to cross at particular cells. Now that you know the rules of the game, here are the properties that I found interesting:
	
\begin{enumerate}
	\item All puzzles are solvable and the solutions are unique
	\item All 2x2 sets of cells have at least 2 colors
\end{enumerate}

\section{Gathering Solutions for Puzzles in the Game}

I wanted utilize the existing solutions to understand the problem better and do some processing on known solutions.

I got the solutions from the game files of the app. Pulled the APK, decompiled it and found the files in the assets.

Understanding the file format.

\section{Generating Puzzles}

There are a few algorithms I would like to explore for generating these puzzles. Some have more obvious advantages over others but it doesn't hurt to try them all. The algorithms are:
\begin{enumerate}
  \item Wave Function Collapse
  \item Modified Maze Generation algorithms
  \item Hamiltonian paths
  \item Brute Force
\end{enumerate}

Wave function collapse works by being provided with certain constraints and it randomly propagates those constraints to produce randomized outputs that doesn't contradict the provided constraints.
Since our problem has many of these rules that the puzzle must follow, this seems like a very promising approach. 
Maze Generation algorithms seem interesting because maze generation algorithms have to fully cover a grid. 
Sadly, mazes usually have all their "hallways" connected to each other. Similar to maze generation algorithms, hamiltonian paths seem
interesting because they fully cover the grid. However, they have the same problem as maze generation algorithms where each cell is guaranteed connected to all other cells.

\subsection{Wave Function Collapse}

The first part of the wave function collapse algorithm is defining the constraints to use. 
I began with only allowing the types of cells that would be valid for a puzzle:

% Add pictures of these instead of listing them
\begin{itemize}
  \item LEFT
  \item RIGHT
  \item UP
  \item DOWN
  \item UP DOWN (vertical line)
  \item LEFT RIGHT (horizontal line)
  \item LEFT DOWN
  \item RIGHT DOWN
  \item LEFT UP
  \item RIGHT UP
\end{itemize}

This was a good start from what I thought because we want to generate puzzles that only have these cells in them. However, I quickly noticed an issue.
After writing the collapse algorithm I tested it out and could almost never generate a valid puzzle. It would often collapse some cells into a state where no possible connection could be made.

So, some modifications had to be made. I then changed the constraints to allow for any connection on a grid, even if the conenction form was not valid for our puzzle.
These new constraints let me almost always generate a proper filled out grid of cells but the results were still not valid puzzles.
Occasionally it would generate a grid with holes on no possible states being valid even though there should always be one so I am guessing this was a bug in my code.

Now, we have a method to produce a set of connections with endpoints that fully cover a 2d grid. This is pretty close but there is still one step where we have to add in the constraints that only 2 endpoints can be connected and there should be no loops.

\subsection{Modified Maze Generation}

\subsection{Brute Force}



\section{Unique Puzzle Solutions}

\section{Wrinkle-Free Connections}

\end{document}